% Gemini theme
% https://github.com/anishathalye/gemini

\documentclass[final]{beamer}
\usepackage{anyfontsize}

% ====================
% Packages
% ====================

\usepackage{magma}

\usepackage{catalan-drawings}

\renewcommand{\baselinestretch}{1.1} 

\usepackage{lmodern}
\usepackage[T1]{fontenc}
\usepackage[size=a3, orientation=portrait, scale=1.5]{beamerposter}
\usetheme{gemini}
\usecolortheme{gemini}
\usepackage{graphicx}
\usepackage{booktabs}
\usepackage{tikz}
\usepackage{pgfplots}

% ====================
% Lengths
% ====================

% If you have N columns, choose \sepwidth and \colwidth such that
% (N+1)*\sepwidth + N*\colwidth = \paperwidth
\newlength{\sepwidth}
\newlength{\colwidth}
\setlength{\sepwidth}{0.05\paperwidth}
\setlength{\colwidth}{0.425\paperwidth}

\newcommand{\separatorcolumn}{\begin{column}{\sepwidth}\end{column}}

% ====================
% Title
% ====================
\makeatletter
\newcommand\HUGE{\@setfontsize\Huge{45}{47}} 
\makeatother
% {\HUGE\texttt{magma}}
\title{magma : A Library of Universal Catalan Bijections}

\author{Billy Price, Supervisor: Richard Brak}

\institute[shortinst]{~}

% ====================
% Footer (optional)
% ====================

\footercontent{
  % \href{https://www.example.com}{https://www.example.com} \hfill
  Vacation Scholarship Program, 2020 --- The University of Melbourne \hfill
  \href{mailto:billy.price@unimelb.edu.au}{billy.price@unimelb.edu.au}}
% (can be left out to remove footer)

% ====================
% Logo (optional)
% ====================

% use this to include logos on the left and/or right side of the header:
% \logoright{\includegraphics[height=7cm]{logo1.pdf}}
% \logoleft{\includegraphics[height=7cm]{logo2.pdf}}

% ====================
% Body
% ====================

\newcommand{\getTri}[1]{\begin{tikzpicture}[line width=2pt, radius=3pt]\getdata{#1}\triangulations\end{tikzpicture}}
\newcommand{\getSP}[1]{\begin{tikzpicture}[line width=2.5pt, radius=4pt, scale=0.75]\getdata{#1}\staircasePolygons\end{tikzpicture}}
\newcommand{\getCBT}[1]{\scalebox{0.5}{\getdata{#1}\CBTs}}
\newcommand{\getArch}[1]{\begin{tikzpicture}[line width=2pt, radius=4pt, scale=0.75]\getdata{#1}\linkDiagrams\end{tikzpicture}}

  
  

% \definecolor{red}{RGB}{165, 38, 38}


\begin{document}

\begin{frame}[t]
\begin{columns}[t]
\separatorcolumn

\begin{column}{\colwidth}

  {
    \usebeamerfont{caption}
    \begin{tabular}{cccc}
      \shortstack{Complete \\ Binary \\ Trees} & Triangulations & \shortstack{Link \\ Diagrams} & \shortstack{Staircase \\ Polygons} \\
      &&& \\
      % \shortstack{$\mathrm{Norm}=1$ \\ (generator)} & & & \\
      \getCBT{1} & \getTri{1} & \getArch{1} & \getSP{1} \\
      % $\mathrm{Norm}=2$ & & & \\
      \hline
      \getCBT{2} & \getTri{2} & \getArch{2} & \getSP{2} \\
      % $\mathrm{Norm}=3$ & & & \\
      \hline
      \getCBT{3} & \getTri{3} & \getArch{3} & \getSP{3} \\
      \getCBT{4} & \getTri{4} & \getArch{4} & \getSP{4} \\
      % $\mathrm{Norm}=4$ & & & \\
      \hline
      \getCBT{5} & \getTri{5} & \getArch{5} & \getSP{5} \\
      \getCBT{6} & \getTri{6} & \getArch{6} & \getSP{6} \\
      \getCBT{7} & \getTri{7} & \getArch{7} & \getSP{7} \\
      \getCBT{8} & \getTri{8} & \getArch{8} & \getSP{8} \\
      \getCBT{9} & \getTri{9} & \getArch{9} & \getSP{9}
    \end{tabular}
  }

\begin{block}{A larger example - which objects above multiply to give these?}
  \begin{tabular}{ccc}
    \getCBT{10} & \renewcommand{\defaultRadius}{1.5} \getTri{10} & \getSP{10}
  \end{tabular}
  \begin{center}
    \getArch{10}
  \end{center}
\end{block}
    
  % \begin{block}

  %   Some block contents, followed by a diagram, followed by a dummy paragraph.

  %   \begin{figure}
  %     \centering
  %     \begin{tikzpicture}[scale=6]
  %       \draw[step=0.25cm,color=gray] (-1,-1) grid (1,1);
  %       \draw (1,0) -- (0.2,0.2) -- (0,1) -- (-0.2,0.2) -- (-1,0)
  %         -- (-0.2,-0.2) -- (0,-1) -- (0.2,-0.2) -- cycle;
  %     \end{tikzpicture}
  %     \caption{A figure caption.}
  %   \end{figure}

  %   Lorem ipsum dolor sit amet, consectetur adipiscing elit. Morbi ultricies
  %   eget libero ac ullamcorper. Integer et euismod ante. Aenean vestibulum
  %   lobortis augue, ut lobortis turpis rhoncus sed. Proin feugiat nibh a
  %   lacinia dignissim. Proin scelerisque, risus eget tempor fermentum, ex
  %   turpis condimentum urna, quis malesuada sapien arcu eu purus.

  % \end{block}

  % \begin{block}{A block containing a list}

  %   Nam vulputate nunc felis, non condimentum lacus porta ultrices. Nullam sed
  %   sagittis metus. Etiam consectetur gravida urna quis suscipit.

  %   \begin{itemize}
  %     \item \textbf{Mauris tempor} risus nulla, sed ornare
  %     \item \textbf{Libero tincidunt} a duis congue vitae
  %     \item \textbf{Dui ac pretium} morbi justo neque, ullamcorper
  %   \end{itemize}

  %   Eget augue porta, bibendum venenatis tortor.

  % \end{block}

\end{column}

\separatorcolumn

\begin{column}{\colwidth}

  \begin{block}{$1, 1, 2, 5, 14, 42, 132, 429, 1430, 4862\dots$}
    The Catalan Numbers are an integer sequence (A000108) counted by any of 214 combinatorial families described in R. Stanley's \textit{Catalan Numbers}. We would like to understand properties of these families via bijections to other Catalan families, however, it would be infeasible to design $214 \times 213 = 4582$ direct bijections, and composing bijections lacks canonicity.

    Through R. Brak's insight that any Catalan family is just a Free Magma on one generator, we reduce the problem of finding bijections between pairs of families, to finding a unique way of multiplying and factorising objects within a family (just once for each family!). 
  \end{block}

  \begin{alertblock}{\texttt{magma}}
    \texttt{magma} is an open-source library/catalogue of Catalan Families, which provides a universal bijection function between any two Catalan Families. Given two Catalan Families, \texttt{A}, \texttt{B}, we obtain the bijection recursively:
    
    \texttt{
    bij(x) :: A -> B\\
    ~~~if x == A.generator:\\
    ~~~~~return B.generator\\
    ~~~else:\\
    ~~~~~first, second := A.factorise(x)\\
    ~~~~~return B.multiply( bij(first), bij(second) )}

    Therefore, anyone with their own Catalan Family can implement a valid \texttt{multiply} and \texttt{factorise} function and immediately obtain a bijective function to/from any other Catalan Family.
  \end{alertblock}

\end{column}

\separatorcolumn
\end{columns}

  % \begin{tabular*}{\pagewidth}{c@{\extracolsep{\fill}}c@{\extracolsep{\fill}}c@{\extracolsep{\fill}}c@{\extracolsep{\fill}}}
  %   \getCBT{10} & \getTri{10} & \getArch{10} & \getSP{10}
  % \end{tabular*}

  
    
\end{frame}

\end{document}
