% Gemini theme
% https://github.com/anishathalye/gemini

\documentclass[final]{beamer}
\usepackage{anyfontsize}

% ====================
% Packages
% ====================

\usepackage{magma}

\usepackage{catalan-drawings}

\renewcommand{\baselinestretch}{1.2} 

\usepackage{lmodern}
\usepackage[T1]{fontenc}
\usepackage[size=a3, orientation=portrait, scale=1.5]{beamerposter}
\usetheme{gemini}
\usecolortheme{gemini}
\usepackage{graphicx}
\usepackage{booktabs}
\usepackage{tikz}
\usepackage{pgfplots}

% ====================
% Lengths
% ====================

% If you have N columns, choose \sepwidth and \colwidth such that
% (N+1)*\sepwidth + N*\colwidth = \paperwidth
\newlength{\sepwidth}
\newlength{\colwidth}
\setlength{\sepwidth}{0.02\paperwidth}
\setlength{\colwidth}{0.425\paperwidth}

\newcommand{\separatorcolumn}{\begin{column}{\sepwidth}\end{column}}

% ====================
% Title
% ====================
\makeatletter
\newcommand\HUGE{\@setfontsize\Huge{45}{47}} 
\makeatother
% {\HUGE\texttt{magma}}
\title{magma : A Library of Universal Catalan Bijections}

\author{Billy Price, Supervisor: Richard Brak}

\institute[shortinst]{~}

% ====================
% Footer (optional)
% ====================

\footercontent{
  % \href{https://www.example.com}{https://www.example.com} \hfill
  Vacation Scholarship Program, 2020 --- The University of Melbourne \hfill
  \href{mailto:billy.price@unimelb.edu.au}{billy.price@unimelb.edu.au}}
% (can be left out to remove footer)

% ====================
% Logo (optional)
% ====================

% use this to include logos on the left and/or right side of the header:
% \logoright{\includegraphics[height=7cm]{logo1.pdf}}
% \logoleft{\includegraphics[height=7cm]{logo2.pdf}}

% ====================
% Body
% ====================

\newcommand{\getTri}[1]{\begin{tikzpicture}[line width=2pt, radius=3pt]\getdata{#1}\triangulations\end{tikzpicture}}
\newcommand{\getSP}[1]{\begin{tikzpicture}[line width=2.5pt, radius=4pt, scale=0.75]\getdata{#1}\staircasePolygons\end{tikzpicture}}
\newcommand{\getCBT}[1]{\scalebox{0.5}{\getdata{#1}\CBTs}}
\newcommand{\getArch}[1]{\begin{tikzpicture}[line width=2pt, radius=4pt, scale=0.75]\getdata{#1}\linkDiagrams\end{tikzpicture}}

  
  

% \definecolor{red}{RGB}{165, 38, 38}


\begin{document}

\begin{frame}[t]
\begin{columns}[t]


\separatorcolumn


\begin{column}{\colwidth}

  {
    \usebeamerfont{caption}
    \begin{tabular}{cccc}
      \shortstack{Complete \\ Binary \\ Trees} & Triangulations & \shortstack{Staircase \\ Polygons} & \shortstack{Link \\ Diagrams} \\
      &&& \\
      % \shortstack{$\mathrm{Norm}=1$ \\ (generator)} & & & \\
      \getCBT{1} & \getTri{1} & \getSP{1} & \getArch{1} \\
      % $\mathrm{Norm}=2$ & & & \\
      \hline
      \getCBT{2} & \getTri{2} & \getSP{2} & \getArch{2} \\
      % $\mathrm{Norm}=3$ & & & \\
      \hline
      \getCBT{3} & \getTri{3} & \getSP{3} & \getArch{3} \\
      \getCBT{4} & \getTri{4} & \getSP{4} & \getArch{4} \\
      % $\mathrm{Norm}=4$ & & & \\
      \hline
      \getCBT{5} & \getTri{5} & \getSP{5} & \getArch{5} \\
      \getCBT{6} & \getTri{6} & \getSP{6} & \getArch{6} \\
      \getCBT{7} & \getTri{7} & \getSP{7} & \getArch{7} \\
      \getCBT{8} & \getTri{8} & \getSP{8} & \getArch{8} \\
      \getCBT{9} & \getTri{9} & \getSP{9} & \getArch{9}
    \end{tabular}
  }

\begin{block}{A larger example - what does the corresponding link diagram look like?}
  \begin{tabular}{ccc}
    \getCBT{10} & \renewcommand{\defaultRadius}{1.5} \getTri{10} & \getSP{10}
  \end{tabular}
  % \begin{center}
  %   \getArch{10}
  % \end{center}
\end{block}

\end{column}

\separatorcolumn

\begin{column}{\colwidth}

  \begin{block}{$1, 1, 2, 5, 14, 42, 132, 429, 1430, 4862\dots$}
    The Catalan Numbers are an integer sequence (A000108) counted by any of 214 combinatorial families described in R. Stanley's \textit{Catalan Numbers}. We would like to understand properties of these families via bijections to other Catalan families, however, it would be infeasible to design $214 \times 213 = 4582$ direct bijections, and composing bijections lacks canonicity.

    Through R. Brak's insight that any Catalan family is just a Free Magma on one generator, we reduce the problem of finding bijections between pairs of families, to finding a unique way of multiplying and factorising objects within a family (just once for each family!). 
  \end{block}

  \begin{alertblock}{\texttt{magma}}
    \texttt{magma} is an open-source library/catalogue of Catalan Families, which provides a universal bijection function between any two Catalan Families. Given two Catalan Families, \texttt{A}, \texttt{B}, we obtain the bijection, \texttt{bij :: A -> B} recursively:
    
    \texttt{
    def bij(x):\\
    ~~~if x == A.generator:\\
    ~~~~~return B.generator\\
    ~~~else:\\
    ~~~~~first, second := A.factorise(x)\\
    ~~~~~return B.multiply( bij(first), bij(second) )}

    Therefore, anyone with their own Catalan Family can implement a valid \texttt{multiply} and \texttt{factorise} function and immediately obtain a bijective function to/from any other Catalan Family.
  \end{alertblock}

  \begin{block}{Representing Families}
    A central challenge to this project was deciding on the best way to represent each family's objects in the code, especially as many Catalan Families are geometric in nature. As is this re
  \end{block}

\end{column}

\separatorcolumn

\end{columns}

  % \begin{tabular*}{\pagewidth}{c@{\extracolsep{\fill}}c@{\extracolsep{\fill}}c@{\extracolsep{\fill}}c@{\extracolsep{\fill}}}
  %   \getCBT{10} & \getTri{10} & \getArch{10} & \getSP{10}
  % \end{tabular*}

  
    
\end{frame}

\end{document}
